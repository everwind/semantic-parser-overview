\documentclass{beamer}

%\usepackage{utopia}
\usetheme{Madrid}
\usecolortheme{default} % or use {default}

% parse tree
\usepackage[nocenter]{qtree}
\usepackage{tree-dvips}
\usepackage{amsmath}

%\usecolortheme{default}

\title[]{Semantic Parsing Methods}
\subtitle{An Overview}
\author {Xiang Zhang}
%\institute {
%    NLPR\\
%    Institute of Automation
%}
\date{2016.08}
%\logo{\includegraphics[height=1.5cm]{lion-logo.png}}

%------------------------------------------------------------
\AtBeginSection {
    \begin{frame}
        \frametitle{Agenda}
        
        % current subsection / other subsections of current section / other subsections
        \tableofcontents[sectionstyle=show/shaded,subsectionstyle=hide/hide/hide]
    \end{frame}
}
\AtBeginSubsection {
    \begin{frame}
        \frametitle{Agenda}
        
        % current subsection / other subsections of current section / other subsections
        \tableofcontents[subsectionstyle=show/shaded/hide]
    \end{frame}
}
%------------------------------------------------------------

\begin{document}

\frame{\titlepage}

%---------------------------------------------------------
%This block of code is for the table of contents after
%the title page
\begin{frame}
\frametitle{Agenda}
\tableofcontents[hideallsubsections]
\end{frame}
%---------------------------------------------------------


\section{Semantics}

\begin{frame}
    \frametitle{Background}

    When it comes to the understanding of natural language sentences, NLP researchers 
    solve it in various granularities.

    These tasks differ in the amount of information they use.

    \begin{itemize}
        \item <1-> Information Extraction (less informative) \\
            \begin{center}
                \emph{is\_a(Obama, PRESIDENT)}
            \end{center}

        \item <2-> Summarization (modestly informative) \\
            \begin{center}
            \emph{Obama wins.}
            \end{center}

        \item <3-> Semantic Parsing (exact matching) \\
            \begin{center}
            $\exists e . beat(e) \wedge Sub(e, Obama) \wedge Obj(e, Romney)$
            \end{center}
            
    \end{itemize}

    \uncover<4->{\begin{block}{Caveat}
        \emph{Semantic} here is more of \emph{composition} than telling apart
        from \emph{word senses}.
    \end{block}}
\end{frame}

\begin{frame}
    \frametitle{Semantic Parsing Task}

    The key task of semantic parsing is to find an $f$ such that

    \[
        f: Sentence \to LogicForm
    \]

    \pause

    Generally, there are 3 aspects a semantic parser need take into consideration:

    \begin{itemize}
        \item Modelling: how to represent a logic form
        \item Parsing: design a grammar and parsing algorithm
        \item Learning: use supervision to fix parameters
    \end{itemize}

\end{frame}

\begin{frame}
    \frametitle{Logic Form from Example}

    \begin{itemize}

        \item <2->
            Brutus stabs Caesar. \\
            stab(Brutus, Caesar) \structure{predicate}

        \item <3->
            Brutus stabs Caesar with a knife. \\
            stab(Brutus, Caesar, \alert{knife}) \structure{n-ary predicate}

        \item <4->
            Brutus stabs Caesar in the agora. \\
            stab(Brutus, Caesar, \alert{agora}) \structure{ambiguous predicate}

        \item <5->
            Brutus stabs Caesar in the agora with a knife. \\
            stab(Brutus, Caesar) \& \alert{with}(knife) \& \alert{in}(agora)
            \structure{move adjunct apart}

    \end{itemize}

\end{frame}

\begin{frame}
    \frametitle{Logic Form from Example}

    \begin{itemize}
        \item <1-> Brutus stabs Caesar in the agora with a knife. \\
            stab(Brutus, Caesar) \& with(knife) \& in(agora)

        \item <2-> Brutus stabs Caesar with a knife in the agora and twisted it hard. \\
            stab(Brutus, Caesar) \& with(knife) \& in(agora) \& twist(Brutus, knife) \& hard

    \end{itemize}

    \uncover <3-> {
        The standard predicate calculus has problems.

        \begin{itemize}
            \item unable to refer to predicates
            \item natural language are flexible in the number of arguments
                \begin{itemize}
                    \item Pass the axe.
                    \item Pass \alert{me} the axe.
                \end{itemize}
        \end{itemize}
    }

\end{frame}

\begin{frame}
    \frametitle{Davidsonian Representation}

    Semantic is characterized in \emph{events}.
    We don't know an event beforehand, thus we \alert{existentially quantify} it.

    \begin{itemize}

        \item Brutus stabs Caesar with a knife in the agora and twisted it hard.
            \begin{gather*}
                \exists e . stab(e, Brutus, Caesar) \wedge with(e, knife) \wedge in(e, agora)\\
                \wedge (\exists e' . twist(e', Brutus, knife) \wedge hard(e'))
            \end{gather*}

        \item Caesar is stabbed.
            \[
                \exists x \exists e . stab(e, x, Caesar)
            \]

            Missing arguments are left with \alert{placeholders}.
    \end{itemize}

\end{frame}

\begin{frame}

    \frametitle{neo-Davidsonian Representation}


\end{frame}

\section{Parsing}

\begin{frame}
    empty
\end{frame}

\section{Sample Slides}

\begin{frame}
    \frametitle{Sample frame title}
    This is a text in second frame. For the sake of showing an example.
    \[
        x^2 + y^2 = 1
    \]

    \begin{itemize}
        \item <1-> Text visible on slide 1
        \item <2-> Text visible on slide 2
        \item <3> Text visible on slides 3
        \item <4-> Text visible on slide 4
    \end{itemize}

\end{frame}

\begin{frame}
In this slide \pause

the text will be partially visible \pause

And finally everything will be there
\end{frame}

%---------------------------------------------------------
%Highlighting text
\begin{frame}
    \frametitle{Sample frame title}

    In this slide, some important text will be
    \alert{highlighted} beause it's important.
    Please, don't abuse it.

    \begin{block}{Remark}
    Sample text
    \end{block}

    \begin{alertblock}{Important theorem}
    Sample text in red box
    \end{alertblock}

    \begin{examples}
    Sample text in green box. "Examples" is fixed as block title.
    \end{examples}
\end{frame}
%---------------------------------------------------------

%---------------------------------------------------------
%Two columns
\begin{frame}
    \frametitle{Two-column slide}

    \begin{columns}

        \column{0.5\textwidth}
            This is a text in first column.
            $$E=mc^2$$
            \begin{itemize}
                \item First item
                \item Second item
            \end{itemize}

        \column{0.5\textwidth}
            This text will be in the second column
            and on a second tought this is a nice looking
            layout in some cases.

        \Tree [.S This [.VP [.V is ] \qroof{a simple tree}.NP ] ]
    \end{columns}
\end{frame}
%---------------------------------------------------------

\end{document}
